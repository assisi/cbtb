\documentclass{article}
\usepackage[utf8]{inputenc}
\usepackage{minted}
\usepackage{amsmath}

\newcommand{\scare}[1]{``#1''} 

\title{Collective behaviour Indices}

\date{}%{\today}

\begin{document}
\maketitle

\noindent Some metrics aimed at quantifying bee group behaviour are implemented in the `cbtb` python package.  This document describes these metrics and provides code examples. 

\section{(Group) Volatility index}

In a collective binary choice assay:

\paragraph*{Intuition:} Quantify the number of \scare{total population swings}, i.e., the number of times a group moves entirely from one decision to the other decision, per time unit.  The metric is fractional, i.e., it can count partial group moves.

\paragraph{Computation}
Given $n$ bees and the counts of bees on the left/right side at timestep $t$ are given by $L(t), \in [0,n]$; $R(t), \in [0,n]$, s.t. $L(t) + R(t) = n, \quad \forall t$, we compute the volatility score $\upsilon(t)$ for one timestep $t$

\begin{equation}
	\upsilon(t) = \frac{\|L(t) - L(t+1) \|}{n},
\end{equation}
and for a whole experiment / analysis window,
\begin{equation}
	V = \frac{r}{t_{max}}\sum_{t=0}^{t_{max}-1} \upsilon(t),
\end{equation}
where $r$ is the rate of samples per analysis window, e.g., if the volatility per minute is desired, and the analysis timesteps are  1$\,$s apart, set $r=60$.

\paragraph{code}
\inputminted{python}{volatility_index_eg.py}

\clearpage
\section{Collective decision-making index}

\paragraph*{Intuition:} This metric is based on a time budget of how much time a group spends in a significantly unlikely aggregation.  For a binary decision in an unbiased environment, we define the null model as the binomial distribution, such that a large number of the group need to have made the same decision (shown through their position) for the configuration to be considered as \scare{significantly unlikely}.

The binomial distribution tells us what would happen if all the actors moved randomly, and independently from one another.

For a confidence level of $p=0.05$, we find that among groups of $n=12$, aggregations of $a\geq 10$ actors on the same side are considered significant.


\paragraph{Computation}

Given $n$ bees and the counts of bees on the left/right side at timestep $t$ are given by $L(t), \in [0,n]$; $R(t), \in [0,n]$, s.t. $L(t) + R(t) = n, \quad \forall t$, and a threshold for significant decision $d$ that is derived from the binomial distribution.

We can compute the majority side $M(t)$ for each timestep as
\begin{equation}
	M(t) = \max{(L(t), R(t))}.
\end{equation}
Then we simply produce a binary value $c$, for each timestep indicating whether the population has significantly decided or not,
\begin{equation}
c(t) = 
\begin{cases}
 & 1 \text{  if } M(t) \geq d, \\ 
 & 0  \text{  otherwise.}
\end{cases}
\end{equation}
Finally, we produce a mean value of these per-step decisions such to give a fractional time budget, in the range $[0, 1]$,
\begin{equation}
    C = \frac{1}{t_{max}}\sum_{t=0}^{t_{max}} c(t).
\end{equation}


\paragraph{Code}
It should be noted that the implementation below uses an upper and a lower threshold in order to compute the value for $c(t)$ directly from $L$ rather than first computing $M$.  I have no idea which is computationally more efficient.

\inputminted{python}{cdi_eg.py}


\end{document}
